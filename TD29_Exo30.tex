\documentclass[a4paper,10pt]{article}

\usepackage[utf8]{inputenc}
\usepackage[french]{babel}
\usepackage[T1]{fontenc}
\usepackage{amsmath}
\usepackage{amsfonts}
\usepackage{amssymb}
\usepackage{graphicx}
\usepackage{tikz}
\usepackage{fourier}
\usepackage{MnSymbol,wasysym}
\begin{document}
\fbox{36} $\heartsuit$ Soit f un endomorphisme de (E, +, .). Montrons que : Ker($f^0$) $\subset$ Ker($f^1$) $\subset$  Ker($f^2$) $\subset$  Ker($f^3$)..

\textit{Mais d'où ça sort ça, voyons? Ce n'est pas bien dur en vrai :}
Avec les règles de la formule : "CASSE TOI" ou "CAH SOH TOA" (pardon) :\\
$\frownie{}$\\ 

\begin{tikzpicture}
\node (a) at (-0.5,1) {a} ;
\draw (0,0) -- (0,2) ;
\node (a) at (1,1.5) {b} ;
\draw (0,2) -- (2,0) ;
\node (a) at (1,-0.5) {c} ;
\draw (0,0) -- (2,0);
\draw[thick] (0.0,0.1) -- (0.1,0.1);
\draw[thick] (0.1,0.1) -- (0.1,0);
\draw (1.5,0) arc (180:155:1) node {$\theta$}; 
\end{tikzpicture}\\


\parindent=2cm a =?\\
\parindent=2cm b =?\\
\parindent=2cm c =?\\

\parindent=0cm\text{On a} $\tan(\arctan(x))= x = \frac{x}{1}$\\

\text{de plus} $\tan(\theta) = \frac{\text{opposé}}{\text{adjacent}} = \frac{a}{c}$\\
En prenant $\arctan(x)=\theta:$\\
on a l'égalité : $\frac{a}{c} = \frac{x}{1}$ \\
on identifie :\\
\parindent=2cm a=x\\
\parindent=2cm c=1\\

\parindent=0cm \text{Pour avoir b, on sait que par le théorème de Pythagore:} \\
\parindent=2cm $b^2 = a^2 + c^2 = x^2 + 1 $\\
\parindent=2cm $b = \sqrt{x^2+1}$\\

\parindent=0cm \text{De la même manière :} $\sin(\theta) =\frac{\text{opposé}}{\text{hypo}}  = \frac{a}{b} = \frac{x}{\sqrt{x^2+1}} = \sin(\arctan(x))$\\

$\smiley{}$\\
\text{Revenons au calcul intégral :}\\
\text{On pose t=} $\sqrt{1+x^2}$: \\
ainsi x=$\sqrt{t^2-1}$ \\
\text{et on a : dx =} $\frac{2t}{2\cdot \sqrt{t^2-1}} \cdot dt$ = $\frac{t}{\sqrt{t^2-1}} \cdot dt$ \text{ après simplification par 2}\\

\text{Ainsi I =} $\int_0^1 \sin(\arctan(x))dx$ = $\int_0^1\frac{x}{\sqrt{x^2+1}} \cdot dx$ =  $\int\frac{t}{\sqrt{t^2-1}} \cdot \frac{\sqrt{t^2-1}}{t} \cdot dt$ = $\int 1 \cdot dt$ \\
\text{Tiens, tiens, une formule facile à calculer : I = }$\left[ t \ \right]$ \\
\text{On remplace par ce qu'on a posé normalement pour revenir aux bornes qu'on connait :} I = $\left[ \sqrt{1+x^2} \ \right]_0^1 = \sqrt{2}-1$ \\

\text{Est-ce que :} $\left( \frac{n^2+n+1}{n^2-n+1} \right)^n$ \text{ peut s'écrire sous la forme} $\left( 1+ \frac{2}{n} \right)^n$ \text{comme on sait que cette dernière tend vers} $e^2$
MERCI A TOI DUC DE M'AVOIR APPRIS A CODER EN \LaTeX{}.
\end{document}
