\documentclass[a4paper,10pt]{article}

\usepackage[utf8]{inputenc}
\usepackage[french]{babel}
\usepackage[T1]{fontenc}
\usepackage{amsmath}
\usepackage{amsfonts}
\usepackage{amssymb}
\usepackage{graphicx}
\usepackage{tikz}
\usepackage{fourier}
\usepackage{MnSymbol,wasysym}
\usepackage{Fancybox}
\begin{document}
\fbox{36} $\heartsuit$ Soit f un endomorphisme de (E, +, .). Montrons que : Ker($f^0$) $\subset$ Ker($f^1$) $\subset$  Ker($f^2$) $\subset$  Ker($f^3$)..\\

\text{En effet, on a pour g et f, endomorphismes de (E,+,.) : Ker(f)} $\subset$ Ker(g$\circ$f) qui est l'implication : \\

\begin{center}
\ovalbox{(f($\vec{a}$)=$\overrightarrow{0_F}$) $\Rightarrow$ (g(f($\vec{a}$)) = $\overrightarrow{0_G}$)}\\
\end{center}
Ceci est bien la définition de l'inclusion A $\subset$ B :
\ovalbox{$\forall$ x (x $\in$ A $\Rightarrow$ x $\in$ B)}\\

Ainsi, en prenant f et g, les bons endomorphismes, on arrive bien à :\\ \cornersize{2}
\setlength{\fboxsep}{2.5 mm}
\setlength{\fboxrule}{1 mm}
\center \fbox{\textcolor{red}{Ker($f^0$) $\subset$ Ker($f^1$) $\subset$  Ker($f^2$) $\subset$  Ker($f^3$)..}}
\end{document}