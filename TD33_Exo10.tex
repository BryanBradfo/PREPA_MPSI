\documentclass[a4paper,10pt]{article}

\usepackage[utf8]{inputenc}
\usepackage[french]{babel}
\usepackage[T1]{fontenc}
\usepackage{amsmath}
\usepackage{amsfonts}
\usepackage{mathtools}
\usepackage{amssymb}
\usepackage{graphicx}
\usepackage{tikz}
\usepackage{fourier}
\usepackage{MnSymbol,wasysym}
\usepackage{fancybox}
\usepackage{systeme}
\usepackage{enumerate}
\begin{document}
\fbox{10}  I=$\int_0^{\frac{\pi}{2}}$ $\frac{\cos(\theta)}{\cos^3(\theta)+\sin^3(\theta)}$ $\cdot$ d$\theta$ et J=$\int_0^{\frac{\pi}{2}}$ $\frac{\sin(\theta)}{\cos^3(\theta)+\sin^3(\theta)}$ $\cdot$ d$\theta$ \\

Pour cet exercice, j'ai envisagé un raisonnement plus calculatoire, d'autres approches sont envisageables, comme peut-être passer par I+J et I-J (et J-I) si jamais ces intégrales sont plus faciles à calculer, pour trouver I et J : \\

Cette méthode permettra de revoir certaines méthodes et on jonglera avec quelques astuces : 

J=$\int_0^{\frac{\pi}{2}}$ $\frac{\sin(\theta)}{\cos^3(\theta)+\sin^3(\theta)}$ $\cdot$ d$\theta$ \\ 

Pour l'existence, le dénominateur n'est pas nul le long des bornes de l'intervalle. \\

Alors, en multipliant numérateur et dénominateur par $\frac{1}{sin^3(\theta)}$ \\

On obtient : J = $\int_0^{\frac{\pi}{2}}$ $\frac{\frac{1}{\sin^2(\theta)}}{\cot^3(\theta)+1}$ $\cdot$ d$\theta$ \\

Rappel qui ne fait pas de mal : cot = $\frac{\cos}{\sin}$ \\

J'effectue un changement de variable en posant : $\theta = \tan(u)$ soit d$\theta = (1 + \tan^2(u)) \cdot du = \frac{du}{\cos^2(u)}$ \\
soit u = $\cot(\theta)$ soit du = - $\frac{d\theta}{\sin^2(\theta}$

Ainsi, on a : $\int - \frac{du}{u^3-1} = - \int \frac{du}{(u-1)(u^2+u+1)} = - \int \frac{A}{u-1} + \frac{Bu+C}{u^2+u+1}$ \\
On a du Bu+C au numérateur car dénominateur est un polynôme à discriminant négatif \\

Après identification : A = $\frac{1}{3}$ ; B = -$\frac{1}{3}$ ; C = -$\frac{2}{3}$, pardon, vous voulez vraiment que je la fasse ? Allons-y : \\

Du coup, il y aurait égalité : $\frac{du}{(u-1)(u^2+u+1)} = \frac{A}{u-1} + \frac{Bu+C}{u^2+u+1}$ \\

\parindent=0cm
1 = A $\cdot$ ($u^2$+u+1) + (Bu+C) $\cdot$ (u-1) \\
1 = $u^2$ $\cdot$ (A+B) + u $\cdot$ (A-B+C) + (A-C)

\parindent=0.6cm
$\systeme{A+B=0,A-B+C=0, A-C=1}$ \\
$\Leftrightarrow$ $\systeme{A=-B,-2B+A-1=0, C=A-1}$ \\
$\Leftrightarrow$ $\systeme{A=\frac{1}{3}, B=-\frac{1}{3}, C=-\frac{2}{3}}$\\

Bon, repassons au calcul : \\
J=-(-$\frac{1}{3} \int \frac{du}{u-1} - \frac{1}{6} \int \frac{2u+1}{u^2+u+1} \cdot du - \frac{1}{2} \int \frac{du}{a(u+\frac{1}{2})^2 + \frac{3}{4}}$) \\
J= -(-$\frac{1}{3} \cdot \left[ ln(u-1) \right] - \frac{1}{6} \cdot \left[ ln(u^2+u+1) \right] - \frac{1}{2} \frac{2}{\sqrt{3}} \left[ arctan(\frac{u+\frac{1}{2}}{\frac{\sqrt{3}}{2}}) \right] $\\
Comme u = cot($\theta$) et $\theta$ constituait les bornes de l'intégrale, on obtient :  \\
\cornersize{2}
\setlength{\fboxsep}{2.5 mm}
\setlength{\fboxrule}{1 mm}
 
\fbox{J= -(-$\frac{1}{3} \cdot \left[ ln(\cot(\theta)-1) \right]_0^{\frac{\pi}{2}} - \frac{1}{6} \cdot \left[ ln(cot(\theta)^2+cot(\theta)+1) \right]_0^{\frac{\pi}{2}} - \frac{1}{2} \frac{2}{\sqrt{3}} \left[ arctan(\frac{cot(\theta)+\frac{1}{2}}{\frac{\sqrt{3}}{2}}) \right]_0^{\frac{\pi}{2}} $}\\ \\ \\
\parindent=0cm
Pour I=$\int_0^{\frac{\pi}{2}}$ $\frac{\cos(\theta)}{\cos^3(\theta)+\sin^3(\theta)}$ $\cdot$ d$\theta$, on multiplie par $\frac{1}{cos(\theta)^3}$, on obtient : \\ \\
I = $\int \frac{\frac{1}{cos(\theta)^2}}{1+tan(\theta)^3} \cdot d\theta.$ \\Ce sera la même chose, mais en posant : \\ \\ u = tan($\theta$) soit du = $\frac{d\theta}{cos(\theta)^2}$ \\

Ainsi, on obtient I = $\int \frac{du}{u^3+1}$ \\
 
On refait la même décomposition d'éléments simples : 

I = $\int \frac{du}{(u+1)(u^2-u+1)} = \int (\frac{A}{u+1} + \frac{Bu+C}{u^2-u+1}) \cdot du$

On a par la méthode des pôles :  A = $\frac{1}{3}$ ; B = -$\frac{1}{3}$ ; C=$\frac{2}{3}$

\parindent=0cm
I = $\frac{1}{3} \int \frac{du}{u+1} - \frac{1}{3} \int \frac{u-2}{u^2-u+1} \cdot du$ \\
= $\frac{1}{3} \int \frac{du}{u+1} - \frac{1}{3} \cdot \frac{1}{2} \int \frac{2u-4}{u^2-u+1} \cdot du$ \\
=$ \frac{1}{3} \int \frac{du}{u+1} - \frac{1}{3} \cdot \frac{1}{2} \int \frac{2u-1-3}{u^2-u+1} \cdot du$ \\
= $\frac{1}{3} \int \frac{du}{u+1} - \frac{1}{3} \cdot \frac{1}{2} \int \frac{2u-1}{u^2-u+1} \cdot du - \frac{1}{3} \cdot \frac{3}{2} \int \frac{du}{u^2-u+1}$\\
=$\frac{1}{3} \cdot \left[ ln(u+1) \right] - \frac{1}{6} \cdot \left[ ln(u^2-u+1) \right] + \frac{1}{2} \frac{4}{3} \int \frac{du}{(\frac{2u-1}{\sqrt{3}})^2+1}  $ \\
=$\frac{1}{3} \cdot \left[ ln(u+1) \right] - \frac{1}{6} \cdot \left[ ln(u^2-u+1) \right] + \frac{2}{3} \frac{\sqrt{3}}{2} \left[ arctan(\frac{2u-1}{\sqrt{3}}) \right] $ \\
=$\frac{1}{3} \cdot \left[ ln(u+1) \right] - \frac{1}{6} \cdot \left[ ln(u^2-u+1) \right] + \frac{1}{\sqrt{3}} \left[ arctan(\frac{2u-1}{\sqrt{3}}) \right] $ \\
Comme u = $\tan(\theta)$, on a ainsi 
\begin{center}
\fbox{I=$\frac{1}{3} \cdot \left[ ln(\tan(\theta)+1) \right]_0^{\frac{\pi}{2}} - \frac{1}{6} \cdot \left[ ln(\tan(\theta)^2-\tan(\theta)+1) \right]_0^{\frac{\pi}{2}} + \frac{1}{\sqrt{3}} \left[ arctan(\frac{2\tan(\theta)-1}{\sqrt{3}}) \right]_0^{\frac{\pi}{2}} $}
\end{center}


A la suite, après le conseil de notre coach, pour pas que cela n'explose en $\frac{\pi}{2}$ pour les tangentes, réunissons les logarithmes en un seul :\\
I=$\frac{2}{6} \cdot \left[ ln(\tan(\theta)+1) \right]_0^{\frac{\pi}{2}} - \frac{1}{6} \cdot \left[ ln(\tan(\theta)^2-\tan(\theta)+1) \right]_0^{\frac{\pi}{2}} + \frac{1}{\sqrt{3}} \left[ arctan(\frac{2\tan(\theta)-1}{\sqrt{3}}) \right]_0^{\frac{\pi}{2}} $\\
I=$\frac{1}{6} \cdot \left[ 2 \cdot ln(\tan(\theta)+1) \right]_0^{\frac{\pi}{2}} - \frac{1}{6} \cdot \left[ ln(\tan(\theta)^2-\tan(\theta)+1) \right]_0^{\frac{\pi}{2}} + \frac{1}{\sqrt{3}} \left[ arctan(\frac{2\tan(\theta)-1}{\sqrt{3}}) \right]_0^{\frac{\pi}{2}} $\\
I=$\frac{1}{6} \cdot \left[  ln((\tan(\theta)+1)^2) \right]_0^{\frac{\pi}{2}} - \frac{1}{6} \cdot \left[ ln(\tan(\theta)^2-\tan(\theta)+1) \right]_0^{\frac{\pi}{2}} + \frac{1}{\sqrt{3}} \left[ arctan(\frac{2\tan(\theta)-1}{\sqrt{3}}) \right]_0^{\frac{\pi}{2}} $\\
I=$\frac{1}{6} \cdot \left[  ln(\tan(\theta)^2+2 \cdot \tan(\theta)+1) \right]_0^{\frac{\pi}{2}} - \frac{1}{6} \cdot \left[ ln(\tan(\theta)^2-\tan(\theta)+1) \right]_0^{\frac{\pi}{2}} + \frac{1}{\sqrt{3}} \left[ arctan(\frac{2\tan(\theta)-1}{\sqrt{3}}) \right]_0^{\frac{\pi}{2}} $\\
I=$\frac{1}{6} \cdot \left[ ln(\frac{\tan(\theta)^2+2 \cdot \tan(\theta)+1}{\tan(\theta)^2-\tan(\theta)+1}) \right]_0^{\frac{\pi}{2}} + \frac{1}{\sqrt{3}} \left[ arctan(\frac{2\tan(\theta)-1}{\sqrt{3}}) \right]_0^{\frac{\pi}{2}} $\\
I=$\frac{1}{6} \cdot \left[ ln(\frac{\frac{\sin(\theta)}{\cos(\theta)}^2+2 \cdot \frac{\sin(\theta)}{\cos(\theta)}+1}{\frac{\sin(\theta)}{\cos(\theta)}^2-\frac{\sin(\theta)}{\cos(\theta)}+1}) \right]_0^{\frac{\pi}{2}} + \frac{1}{\sqrt{3}} \left[ arctan(\frac{2\tan(\theta)-1}{\sqrt{3}}) \right]_0^{\frac{\pi}{2}} $\\
I=$\frac{1}{6} \cdot \left[ ln(\frac{\sin(\theta)^2+2 \cdot \sin(\theta) \cdot \cos(\theta)+ \cos(\theta)^2}{\sin(\theta)^2-\sin(\theta)\cos(\theta)+\cos(\theta)^2}) \right]_0^{\frac{\pi}{2}} + \frac{1}{\sqrt{3}} \left[ arctan(\frac{2\tan(\theta)-1}{\sqrt{3}}) \right]_0^{\frac{\pi}{2}} $\\
I=$\frac{1}{6} \cdot \left[ ln(\frac{\sin(\theta)^2+ sin(2 \cdot \theta)+ \cos(\theta)^2}{\sin(\theta)^2-\sin(\theta)\cos(\theta)+\cos(\theta)^2}) \right]_0^{\frac{\pi}{2}} + \frac{1}{\sqrt{3}} \left[ arctan(\frac{2\tan(\theta)-1}{\sqrt{3}}) \right]_0^{\frac{\pi}{2}} $\\
I=$\frac{1}{6} \cdot \left[ ln(\frac{1 + sin(2 \cdot \theta)}{1-\sin(\theta)\cos(\theta)}) \right]_0^{\frac{\pi}{2}} + \frac{1}{\sqrt{3}} \left[ arctan(\frac{2\tan(\theta)-1}{\sqrt{3}}) \right]_0^{\frac{\pi}{2}} $\\
I=$\frac{1}{6} \cdot \left[ ln(\frac{1 + \sin(2 \cdot \theta)}{1-\frac{1}{2} \cdot sin(2 \cdot \theta)}) \right]_0^{\frac{\pi}{2}} + \frac{1}{\sqrt{3}} \left[ arctan(\frac{2\tan(\theta)-1}{\sqrt{3}}) \right]_0^{\frac{\pi}{2}} $\\
Comme le terme : $ln(\frac{1 + \sin(2 \cdot \theta)}{1-\frac{1}{2} \cdot sin(2 \cdot \theta)})$ tend vers 1, au ln, cela faut 0..\\
Pour l'arctangente, il suffit de faire la limite en $\frac{\pi}{2}$
On obtient alors : \\
I= $\frac{1}{\sqrt{3}} (\frac{\pi}{2} - (- \frac{\pi}{6}))$\\
I= $\frac{1}{\sqrt{3}} (\frac{2\cdot \pi}{3})$\\

\fbox{I= $\frac{2\cdot \pi}{3 \cdot \sqrt{3}}$}\\

Bon, je vous admets, j'ai calculé I+J = $\frac{4 \cdot \pi}{3 \sqrt{3}}$ (M A I S c'est un prétexte pour ne pas faire la même chose avec J..) \\

Ainsi, -J = I - (I+J) = $\frac{2\cdot \pi}{3 \cdot \sqrt{3}} - \frac{4 \cdot \pi}{3 \sqrt{3}} =-\frac{2 \cdot \pi}{3 \sqrt{3}}$\\
Finalement : \fbox{J  = $\frac{2\cdot \pi}{ 3\cdot \sqrt{3}}$}

Je l'admets, c'est extrêmement calculatoire.. J'ai essayé avec la somme et la différence, et le petit coup de pouce du prof : $\frac{(a+b)}{a^3+b^3}$ Cela se simplifie.. ! \\

Rectification pour la somme (28/06) : \\

Tout d'abord petit rappel : $a^3 + b^3 = (a+b)(a^2-ab+b^2)$ ainsi $\frac{a+b}{a^3 + b^3}=\frac{1}{(a^2-ab+b^2)}$\\ \\
Calculons alors : 
\begin{align*}
\int_0^{\frac{\pi}{2}} \frac{\sin(\theta) + \cos(\theta)}{\sin(\theta)^3 + \cos(\theta)^3} \cdot d\theta &= \int_0^{\frac{\pi}{2}} \frac{1}{\sin(\theta)^2 - \sin(\theta) \cdot \cos(\theta) + \cos(\theta)^2} \cdot d\theta \\
&=\int_0^{\frac{\pi}{2}} \frac{1}{1 - \sin(\theta) \cdot \cos(\theta) } \cdot d\theta \\
&=\int_0^{\frac{\pi}{2}} \frac{1}{1 - \frac{\sin(\theta)}{\cos(\theta)} \cdot \cos(\theta)^2 } \cdot d\theta \\
&=\int_0^{\frac{\pi}{2}} \frac{1}{1 - \tan(\theta) \cdot \cos(\theta)^2 } \cdot d\theta \\
&=\int_0^{\frac{\pi}{2}} \frac{1}{1 - \tan(\theta) \cdot \frac{1}{1+\tan(\theta)^2} } \cdot d\theta \\
\end{align*}

On applique le changement de variable : $t=\tan(\theta)$ ainsi $d\theta = \frac{dt}{1+t^2}$, on obtient alors : \\
\begin{align*}
\int_0^{\frac{\pi}{2}} \frac{\sin(\theta) + \cos(\theta)}{\sin(\theta)^3 + \cos(\theta)^3} \cdot d\theta &=\int_0^{x} \frac{\frac{dt}{1+t^2}}{1 - \frac{t}{1+t^2}}  \\
&=\int_0^{x} \frac{dt}{1-t+t^2} \\
&= \left[ \frac{2}{\sqrt{3}}\cdot arctan(\frac{2}{\sqrt{3}(t-\frac{1}{2})}) \right]_0^x\\
&= \frac{2}{\sqrt{3}}\arctan(\frac{2}{\sqrt{3}}(x-\frac{1}{2}))-\underbrace{\frac{2}{\sqrt{3}}\arctan(-\frac{1}{\sqrt{3}})}_{+\frac{2}{\sqrt{3}} \cdot \frac{\pi}{6}} \\
\end{align*}

Enfin, \\

$\lim{x\to\infty} \frac{2}{\sqrt{3}}\arctan(\frac{2}{\sqrt{3}}(x-\frac{1}{2}))-\underbrace{\frac{2}{\sqrt{3}}\arctan(-\frac{1}{\sqrt{3}})}_{+\frac{2}{\sqrt{3}} \cdot \frac{\pi}{6}} = \frac{2}{\sqrt{3}} \cdot \frac{\pi}{2} + \frac{2}{\sqrt{3}} \cdot \frac{\pi}{6} = \frac{2}{\sqrt{3}} \cdot \frac{2 \cdot \pi}{3} =$ \fbox{$\frac{4\cdot \pi}{3\sqrt{3}}$}


\end{document}